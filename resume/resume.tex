\documentclass[a4paper]{article}
\usepackage[UTF8]{ctex}
\usepackage{enumitem}
\usepackage{ragged2e}
\usepackage{xcolor}
\usepackage{fontawesome}
\usepackage[colorlinks]{hyperref}
\usepackage{geometry}
\geometry{left=1.0cm,right=1.0cm,top=0cm,bottom=1.0cm}
\newenvironment{cvitems}{
    \begin{justify}
    \begin{description}[
        labelwidth=1.8cm,
        leftmargin=2cm,
        % labelsep=0.5cm,
        parsep=0pt]
    
}{
    \end{description}
    \end{justify}
}
\renewcommand\descriptionlabel[1]{\sffamily #1}

\newenvironment{githubs}{
    \begin{list}{\faGithubSquare}{
        \setlength{\topsep}{0pt}
        \setlength{\partopsep}{0pt}
        \setlength{\parsep}{0pt}
        \setlength{\itemsep}{0pt}}
}{
    \end{list}
}
\def\githublink#1{
    \href{https://github.com/LogCreative/#1}{\sffamily #1}
}

\def\section#1{
    \noindent\hskip2cm \textbf{\large #1} \hrulefill
}

\def\rightnote#1{
    \hfill\textcolor{gray}{\emph{#1}}
}

\title{\textbf{李子龙}}
\author{Log Creative}
\date{}
\begin{document}
    \maketitle
    \thispagestyle{empty}
    
    % 教育经历
    \section{教育经历}
    \begin{cvitems}
        \item[本科生] \textbf{上海交通大学} 二年级 \rightnote{上海}
        
        数学科学学院~数学与应用数学\rightnote{2018--2019}

        电子信息与电气工程学院~计算机科学与技术\rightnote{预计 2019--2023}
    \end{cvitems}

    % 技能
    \section{编程项目}
    \begin{cvitems}
        \item[C/C++] 熟悉使用该语言编写一些数据结构与算法。 \rightnote{2017--2021}
        
        \begin{githubs}
            \item \githublink{GraphGenDecomp} 图生成与分解器:使用了 \textsf{FLTK} 界面库与启发式算法。 \rightnote{C++ 2020}
        \end{githubs}

        \item[VB{\scriptsize .NET}/C\#] 熟悉使用 \textsf{.NET Framework} 框架构造 Windows 应用程序。 \rightnote{2014--2020}
        
        \begin{githubs}
            \item \href{https://github.com/SJTU-Art-Center/ACLiveConsole}{\sffamily ACLiveConsole} 多路直播管理软件,界面化管理 Nginx,支持对 bilibili 弹幕进行艺术处理与多路视频的导播平滑切换。\rightnote{C\# 2019--2020}
            \item \githublink{DialogCreator} 一个界面化的对话框生成器,支持多框弹出的艺术效果。 \rightnote{VB{\scriptsize .NET} 2014--2015}
        \end{githubs}

        \item[\LaTeX] 熟悉样式调整的基本知识、使用 \textsf{tikz} 制作示意图以及使用 \textsf{beamer} 制作幻灯片。 \rightnote{2018--2021}
        
        \begin{githubs}
            \item \githublink{SJTUBeamermin} 个人维护的 \textsf{beamer} 幻灯片模板。\faStar\textsf{~53} \rightnote{2021}
            \item \githublink{LaTeXSparkle} 使用 \textsf{HTML5+CSS+JavaScript} 编写的关于 \LaTeX 的博客。\rightnote{2020--2021}
        \end{githubs}

        \item[Python] 了解如何使用 \textsf{pandas} 进行数据处理、\textsf{matplotlib} 可视化以及 \textsf{keras} 机器学习。 \rightnote{2020--2021}
        
        \begin{githubs}
            \item \githublink{AlgAndComplexity} 算法与复杂性大作业使用 \textsf{Gurobi} 求解器进行任务级数据中心调度。\rightnote{2021}
        \end{githubs}

        \item[JavaScript] 了解如何使用 \textsf{Vue.js} 框架编写网页应用。\rightnote{2020--2021}
        
        \begin{githubs}
            \item \githublink{PGFPlotsEdt} 使用 \LaTeX{} 中的 \textsf{PGFPlots} 宏包绘制统计图的网页应用。\rightnote{2020--2021}
            \item \githublink{AutoBeamer} 转换 Markdown 文件为 \LaTeX{} 中的 \textsf{beamer} 代码以编译出幻灯片。\rightnote{2021}
            \item \href{https://github.com/SJTU-Art-Center/frameSVG}{\sffamily frameSVG} 将图像序列转换为 SVG 动画,以向推送中插入更高清的动画。\rightnote{2020}
            \item \githublink{SJTUCanteenDataWidget} 通过 \textsf{JSON} 获取餐厅就餐数据,用于物联网比赛。\rightnote{2020}
        \end{githubs}
        
        \item[Rust Java] 入门使用该语言编写一些小项目。\rightnote{2021}
        
        \begin{githubs}
            \item \githublink{CompilerPrinciple} 编译原理大作业算符优先级分析表。\rightnote{Rust 2021}
            \item \githublink{OS-Projects} 操作系统实验。\rightnote{C Java 2021}
        \end{githubs}
    \end{cvitems}

    % 竞赛经历

    % 社团经历

\end{document}