\documentclass[a4paper]{article}
\usepackage[UTF8]{ctex}
\usepackage{enumitem}
\usepackage{ragged2e}
\usepackage{xcolor}
\usepackage{geometry}
\geometry{left=1.0cm,right=1.0cm,top=0cm,bottom=1.0cm}
\newenvironment{cvitems}{
    \begin{justify}
    \begin{description}[
        labelwidth=1.8cm,
        leftmargin=2cm,
        % labelsep=0.5cm,
        parsep=0pt]
}{
    \end{description}
    \end{justify}
}

\renewcommand\descriptionlabel[1]{\sffamily #1}

\def\section#1{
    \noindent\hskip2cm \textbf{\large #1} \hrulefill
}

\def\rightnote#1{
    \hfill\textcolor{gray}{\emph{#1}}
}

\title{\textbf{李子龙}}
\author{Log Creative}
\date{}
\begin{document}
    \maketitle
    \thispagestyle{empty}
    
    % 教育经历
    \section{教育经历}
    \begin{cvitems}
        \item[本科生] \textbf{上海交通大学} 二年级 \rightnote{上海}
        
        数学科学学院~数学与应用数学\rightnote{2018--2019}

        电子信息与电气工程学院~计算机科学与技术\rightnote{预计 2019--2023}
    \end{cvitems}

    % 技能
    \section{编程项目}
    \begin{cvitems}
        \item[C/C++] 熟悉使用该语言编写一些数据结构与算法。 \rightnote{2017--2021}
        \item[VB{\scriptsize .NET}/C\#] 熟悉使用 \textsf{.NET Framework} 框架构造 Windows 应用程序。 \rightnote{2014--2020}
        \item[\LaTeX] 熟悉样式调整的基本知识、使用 \textsf{tikz} 制作示意图以及使用 \textsf{beamer} 制作幻灯片。 \rightnote{2018--2021}
        \item[Python] 了解如何使用 \textsf{pandas} 进行数据处理、\textsf{matplotlib} 可视化以及 \textsf{keras} 机器学习。 \rightnote{2020--2021}
        \item[JavaScript] 了解如何使用 \textsf{Vue.js} 框架编写网页应用。\rightnote{2020--2021}
        \item[Rust Java] 入门使用其编写一些小项目。\rightnote{2021}
    \end{cvitems}

    % 竞赛经历

    % 社团经历

\end{document}